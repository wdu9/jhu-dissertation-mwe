


\begin{titlepage}
	\title{\textbf{The Macroeconomic Consequences of Unemployment Scarring}  % \\ \textcolor{red}{PRELIMINARY AND INCOMPLETE}
		\author{
	William Du\thanks{ Johns Hopkins University, Department of Economics, 3100 Wyman Park Drive, Baltimore, MD 21211.
E-mail: \href{wdu9@jhu.edu}{wdu9@jhu.edu}. I am deeply indebted to Christopher Carroll, Jonathan Wright, and Francesco Bianchi for their guidance and support in the development of this paper. For their helpful comments, I thank Mateo Vel\'{a}squez-Giraldo, Wonsik Ko, Bence Bard\'{o}czy, Jeongwon Son, John Green, Tao Wang, Laurence Ball, Greg Duffee, Robert Moffitt, Xincheng Qiu, Yucheng Yang, Jeffrey Sun, Sebastian Graves,  Adrian Monninger, Jamie Lenney, and Kyung Woong Koh. Finally, I thank participants at the 2024 IAAE annual conference and 2024 SCE annual conference. }} \\ \text{\begin{large}Johns Hopkins University \end{large}}}
	\date{\today }
	\maketitle
	\begin{abstract}
	\begin{singlespace}
		\noindent  Job loss leaves scars on wages that persist for more than 20 years. Yet, the importance of this well-documented micro fact for macro dynamics remains largely unexplored. This paper argues that these scars are a key determinant of the speed of macroeconomic recovery following recessions. I incorporate human capital into a heterogeneous agent New Keynesian (HANK) model with search and matching frictions. During unemployment, human capital depreciates, leading to lower wages for reemployed workers. Unemployment scarring, mediated by the fraction of temporary versus permanent layoffs, enables the model to capture \textit{both} the sluggish recovery from the Great Recession and the rapid rebound from the COVID Recession. In particular, the presence of scarring reveals the pivotal role that temporary layoffs fulfilled in supporting the swift post-pandemic recovery and in preventing a subsequent permanent rise in inequality. In a counterfactual analysis of the Great Recession, a U.S. fiscal consolidation would have proven substantially less effective at reducing debt-to-GDP as scarring erodes future tax revenues and therefore increases pressure on the fiscal deficit.

  
\end{singlespace}
\medskip
	
	\end{abstract}

\end{titlepage}

