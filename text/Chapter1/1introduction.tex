
% ----------------------------------------------------------------
\section{Introduction}
% ----------------------------------------------------------------
Since the seminal work of \cite{Jacobson1993}, job loss from stable employment has been understood to cause large and persistent earnings losses.\footnote{In the microeconomic literature, these losses apply to workers who have been employed for 3 to 10 years.} On average, these earnings losses are 15$\%$ after 20 years \citep[e.g.][]{DavisVonWachter2011, Huckfeldt2022}, reflect a permanent loss in wages as opposed to hours \citep[e.g.][]{Moore2019, Lachowska2020,Huckfeldt2022}, worsen in recessions \citep{DavisVonWachter2011, Schmieder2023}, and are concentrated among workers who switch into lower paying occupations \citep{Huckfeldt2022}.\footnote{\cite{Huckfeldt2022} and \cite{Fujita2017} document that over $50\%$ of the unemployed switch occupations. } While a growing \textit{microeconomic} literature seeks to explain the origins of these `scars', few \textit{macroeconomic} papers explore whether these scars matter \textit{quantitatively} for business cycle dynamics, fiscal policy, and monetary policy. The thesis of this paper is that these microeconomic scars are a key determinant of the speed of macroeconomic recovery from recessions.


To quantify the macroeconomic role of microeconomic unemployment scarring, I extend a heterogeneous agent New Keynesian (HANK) model with search and matching (SAM) frictions to include human capital dynamics. In the model, households make a consumption/saving decision in the face of unemployment risk and search frictions in the labor market. To account for the empirical fact that only workers who are permanently laid off suffer from scarring \citep{ Fujita2016}, \footnote{A permanent layoff refers to a worker who has been permanently separated from their previous employer.} the model differentiates between permanent layoffs, temporary layoffs, and other types of unemployment. Temporary layoffs can transition to permanent layoffs and, motivated by recent evidence using U.S. that suggests these scars reflect a loss in productivity \footnote{ The current literature suggests that,in the U.S., the these scars are largely due to losses in firm specific human capital. To begin, \cite{Lachowska2020} find that the decline in wages is largely explained by losses in match specific productivity. \cite{Poletaev2008} find that reemployed workers who suffer large wage losses employ substantially different skills in their new job. \cite{Huckfeldt2022} documents that scarring is concentrated among workers who switch into lower paying occupations.}, only households who find reemployment after a permanent layoff spell \textit{may} experience human capital depreciation.\footnote{Unsurprisingly, it has been documented that workers who have experienced a temporary layoff do not suffer long term earnings losses \citep{Fujita2016}.} The model does not capture the sources that lead firms to engage in temporary layoffs. Instead, using the estimates from \cite{Gertler2022}, the unemployment process across different layoff states is calibrated to match \textit{how} each state evolves during recessions.%\footnote{ Unsurprisingly, \cite{Fujita2016} documents that temporary layoffs in the Current Population Survey "earn about the same income as before (job loss)". } \cite{Arash2015} document in Austrian Administrative data that workers who are recalled experience a slight increase in wages. Finally, in a sample of UI recipients from Missouri and Pennsylvania in the 1980s, \cite{Katz1990} find that, on average, there is a $1.4\%$ decline in hourly earnings for workers who are recalled to their previous employer.} 

I begin by showing that when the model accounts for the microeconomic estimates of unemployment scarring from \cite{DavisVonWachter2011}, the resulting decline in macroeconomic activity is sufficiently persistent to validate unemployment scarring as a new microfoundation for hysteresis and endogenous growth. In particular, with scarring, recessions induce a near-permanent decline in output, consumption, and aggregate labor productivity. Furthermore, since these scars arise from a loss of human capital that reduces both labor income and productivity, the persistent decline in macroeconomic activity occurs without a sustained rise in the unemployment rate. In addition, unemployment scarring induces a permanent increase in wage dispersion that results in a lasting increase in income inequality, a result supported by the data but unaccounted for in standard models of hysteresis or endogenous growth. Finally, the near-permanent decline in wages caused by scarring reduces future tax revenues, increasing the pressure that recessions place on the fiscal deficit since losses in tax revenues necessitate a larger increase in debt to sustain government expenditures. 


Having shown that scarring induces large and persistent declines in macroeconomic activity, I then demonstrate that unemployment scarring, when disciplined by the microeconomic evidence, explains a substantial portion of the sluggish recovery from the Great Recession, a challenging feat that can only be accomplished by a model that can generate a decline in income that is more persistent than the increase in the unemployment rate. To do so, I simulate the model to replicate the path of unemployment from 2008 to 2019 and then compare the untargeted paths of consumption and output against the data. The goal of this exercise is to ask, does the model's predicted path of consumption and output, conditional on the unemployment rate, match the data? Without unemployment scarring, the model only accounts for the first year of the sluggish recovery of consumption and output from The Great Recession. With unemployment scarring, the model's untargeted paths of consumption and output replicate the data from 2008 to 2015, highlighting the substantial role of scarring during the Great Recession. In addition, unemployment scarring also allows the model to replicate the untargeted path of hourly labor compensation for the whole simulation period, providing further validation that the role of scarring during and after the Great Recession is being captured. Finally, unemployment scarring enables the model successfully captures the permanent rise in income inequality following the Great Recession ---a result that standard HANK models with search and matching frictions cannot replicate. In those models, income inequality is largely shaped by the path of unemployment, which, during the Great Recession, increased persistently but not permanently. Overall, the model suggests that scarring played a key role in driving the sluggish recovery from the Great Recession, explaining most of the recovery from 2008 to 2015. This result, however, does not rule out other explanations for the slow recovery from the Great Recession. The aim is to emphasize that unemployment scarring was one of the primary channels that drove the sluggish recovery from the Great Recession.

 Although unemployment scarring explains a substantial fraction of the recovery from the Great Recession, it is the model's ability to predict \textit{both} the swift rebound from the COVID Recession and the slow recovery from the Great Recession that validates unemployment scarring as a key determinant of the speed of macroeconomic recovery from recessions. To illustrate this, I repeat the estimation exercise of matching the path of unemployment during and after the COVID recession and then comparing the untargeted paths of consumption, output, and the Gini index for income. I recalibrate the model such that $98.8\%$ of an increase in unemployment is attributed to temporary layoffs, the proportion of the rise in the unemployment rate accounted by temporary layoffs estimated in \cite{Gertler2022}. Naturally, with an enormous proportion of temporary layoffs, micro unemployment scarring does not translate to macro scarring.  As a result, the model is able to replicate the swift rebound in consumption and output observed in the data, along with the transitory increase in the income Gini.\footnote{The simulation exercise implicitly incorporates the macroeconomic impact of the fiscal policy response because the model is made to match the \textit{observed} unemployment rate in the data during and after the pandemic.} 

The model's success in capturing the COVID recession reveals the crucial role that temporary layoffs fulfilled in supporting the swift post-pandemic recovery and in preventing a lasting rise in inequality. Specifically, I show that had the increase in unemployment during the COVID recession been driven primarily by permanent layoffs, GDP would have failed to return to its pre-recession trend and the income Gini index would have exhibited a persistent rise. To illustrate this, I replicate the COVID recession simulation and recalibrate the model to minimize the share of temporary layoffs contributing to the surge in unemployment. In this counterfactual scenario where temporary layoffs account for only $5\%$ of the rise in unemployment, GDP would have settled on a new trend that is a 2$\%$ deviation below the pre-2020 trend and the income Gini index would have permanently increased by 0.2 percentage points. The emphasis on temporary layoffs does not diminish the role of fiscal policy in accelerating the recovery after the Pandemic. In contrast, temporary layoffs likely complemented fiscal policy, supporting the rapid return to the pre-recession trend. In fact, \cite{Gertler2022} find that the \textit{Paycheck Protection Program} increased employment by increasing the likelihood of being recalled during a temporary layoff. Given this paper's insight that temporary layoffs can prevent unemployment scarring from translating into macroeconomic scarring, the \textit{Paycheck Protection Program} likely played a crucial role in supporting a swift recovery through mitigating the effects of unemployment scarring.


The transmission of fiscal policy changes considerably in the presence of unemployment scarring. Contractionary fiscal multipliers are 0.4 to 1.0 larger and rise, instead of fall, with the horizon due to persistent losses in output. Unemployment scarring also shapes the dynamics of debt in response to contractionary fiscal policy. In particular, when the government cuts spending, losses in future tax revenues increase pressure to issue government debt. This increase in debt combined with larger fiscal multipliers can significantly reduce the effectiveness of fiscal policies aimed at sustaining debt. Furthermore, because unemployment scarring induces a near permanent rise in income inequality, this naturally implies that contractionary fiscal policy also leads to a persistent increase in income inequality.

To quantify the effectiveness of fiscal consolidation, I consider a counterfactual where the U.S. engages in a reduction of government transfers during the Great Recession, a policy pursued by a number of European countries during this period. I demonstrate that unemployment scarring leads fiscal consolidation to cause a significant and prolonged contraction in GDP, with only a minimal reduction in debt-to-GDP. In particular, without scars to unemployment, a  $2\%$ of GDP reduction in government transfers lowers debt-to-GDP by $2.4$ percentage points. With scarring, the decline in debt-to-GDP is only $1$ percentage point. In addition, the fall in GDP from this consolidation lasts 3 to 4 years longer because of losses to human capital that stem from unemployment scarring.

Fiscal consolidation, however, is not always ineffective at stabilizing debt-to-GDP. The zero lower bound plays a crucial role in the ineffectiveness of a U.S. fiscal consolidation during the Great Recession. Without the zero lower bound, debt to GDP would fall by $5$ percentage points instead of $1.2$ percentage points. The larger decline in debt-to-GDP stems from the monetary authority's ability to lower the cost of debt that the government faces. On the other hand, the effects of a lower interest rate do little to mitigate the scarring effects of unemployment on output unless the nominal interest rate is kept lower for considerably longer. 

\textbf{Literature Review} This paper's contributions lie at the intersection of several strands of literature. 

%The first strand is the emerging yet small body of literature on the impact of unemployment scarring on business cycle fluctuations and macroeconomic policy. This whole literature consists of \cite{AlvesViolante2023} and \cite{AlvesViolante2024}, both of which explore the macroeconomic implications of unemployment scarring in the transmission of monetary policy. The main contribution of this paper is to demonstrate that unemployment scarring can quantitatively capture \textit{both} the sluggish recovery from the Great Recession and the rapid rebound following the COVID Recession. Furthermore, this paper highlights the macroeconomic role that temporary layoffs served during the pandemic, showing their critical importance in preventing both a sluggish recovery and a permanent increase in income inequality after the COVID Recession. Finally, while this literature focuses on the monetary policy implications of unemployment scarring, my analysis quantifies its role in the transmission of fiscal policy.

The first strand is the nascent literature on the role of unemployment scarring in shaping business cycle dynamics and macroeconomic policy. To date, this literature comprises only of \cite{AlvesViolante2023} and \cite{AlvesViolante2024}, both of which examine how scarring affects the transmission of monetary policy. This paper makes three contributions to this literature. First, it demonstrates that unemployment scarring can simultaneously account for \textit{both} the sluggish recovery after the Great Recession and the rapid rebound following the COVID Recession. Second, it shows that temporary layoffs played a pivotal role during the pandemic by preventing a sluggish and incomplete recovery. This second contribution highlights that the unusually large share of temporary layoffs during the pandemic was a key reason the post-COVID recovery was significantly faster than the recovery from the Great Recession, even after accounting for the unprecedented fiscal stimulus during COVID. Third, this paper quantifies the importance of scarring in the transmission of fiscal policy.

The second is the theoretical literature on endogenous growth and hysteresis that largely emphasizes the role of endogenous innovation and R$\&$D as a micro foundation that explains the sluggish recovery of productivity from past recessions \citep{Comin2006,Queralto2018,Bianchi2019}.  Although unemployment scarring has long been considered as a potential mechanism for the sluggish recoveries from past recessions \citep{Cerra2023}, there is surprisingly little work that captures unemployment scarring in a macroeconomic model of the business cycle. This paper addresses this gap by quantifying the importance of these unemployment scars across past recessions. More interestingly, I show that unemployment scarring is a mechanism for hysteresis that can also explain the swift recovery from the COVID Recession when accounting for the large fraction of temporary layoffs during the pandemic. Finally, papers in the literature have also documented that contractionary monetary policy can have persistent effects on the economy \citep{Queralto2018, Jorda2023}. I show that unemployment scarring is an alternative theoretical mechanism that can explain their results (see appendix \ref{appendix:MP}). 

This paper also relates to the literature that documents that fiscal consolidation during the Great Recession induced large and persistent contractions in output \citep{Jorda2016,FATAS2018,House2020}. Most closely related is the work of \cite{FATAS2018}, who estimate the impact of fiscal consolidation on output in Europe during the Great Recession. They find that, on average, the austerity measures pursued by European countries were `self-defeating'. had persistent and contractionary effects on GDP that lasted for at least 10 years. Further, the authors consider unemployment scarring as a possible explanation for their empirical findings. Overall, the authors conclude that fiscal consolidation was `self-defeating'. This paper complements their work by assessing their conjecture with a macroeconomic model that accounts for the microeconomic evidence on unemployment scarring. I show that fiscal consolidation is ineffective at stabilizing debt-to-GDP and has both contractionary and persistent effects on GDP. 

With regards to the distributional consequences of fiscal consolidation, using a sample of 17 OECD countries over the period 1978-2009,  \cite{Ball2013} show that fiscal consolidation raises income inequality. This paper provides a quantitative basis that confirms their empirical results by demonstrating that in the presence of scarring, fiscal contractions lead to a substantial and permanent increase in the Gini index for income. 


Finally, this paper also contributes to the literature on heterogeneous agent New Keynesian (HANK) models, in particular those with search and matching (SAM) frictions. This HANK and SAM literatures emphasizes the interaction between nominal rigidities, search and matching frictions, and incomplete markets to generate counter-cyclical unemployment risk that amplify business cycle fluctuations \citep{McKay2016, ravn2017job, den2018unemployment}. The first contribution of this paper to this literature is the construction of a HANK and SAM model that can capture the scarring effect of unemployment with the inclusion of human capital. The second contribution, found in appendix \ref{appendix:Urisk}, is that the role of unemployment risk as an amplifier of business cycles is considerable larger in the presence of scarring.

\textbf{Outline} The rest of the paper is as follows. Section 2 presents the model. Section 3 describes the parameterization of the model. Section 4 shows that the model is consistent with the microeconomic estimates of earnings loss following job displacement, Section 5 through 10 presents the results. Section 11 concludes. \\ 