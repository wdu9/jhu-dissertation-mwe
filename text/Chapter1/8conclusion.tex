
% ----------------------------------------------------------------
\section{Conclusion}
% ----------------------------------------------------------------

This paper quantifies the macroeconomic role of a well-documented microeconomic fact, that job loss leads to scars on wages. Incorporating these microeconomic scars into a heterogeneous agent New Keynesian model with search and matching frictions introduces a novel channel that emerges as a key determinant of the speed of macroeconomic recovery from a recession. When estimated to match the microeconomic estimates on scarring, and calibrated to match the fraction of temporary layoffs in each recession, the model is able to quantitatively capture \textit{both} the sluggish recovery from the Great Recession and the swift rebound from the COVID Recession. During a recession, the extent to which micro unemployment scarring translates to macro scarring hinges on the share of temporary layoffs driving the rise in the unemployment rate. In particular, had the majority of layoffs during the COVID Recession been permanent rather than temporary, GDP would not have returned to its pre-2020 trend, even when accounting for the large fiscal response during the pandemic.

In addition, the transmission of fiscal austerity changes considerably in the presence of these scars. Given a reduction in government spending, scarring erodes future tax revenues, increasing pressure on the fiscal deficit. Quantitatively, the decline in debt to GDP from a fiscal consolidation is four times smaller because of unemployment scarring and leads to a near permanent rise in income inequality as scarring increases the dispersion in wages. 


The role of unemployment scarring in business cycle dynamics and macroeconomic policy presents many promising avenues for future research. First, the root causes of these scars remain an active area of research. Incorporating the origins of this microeconomic phenomenon into macroeconomic analysis could offer clearer guidance for designing policies to mitigate scarring. Additionally, the connection between unemployment scarring and sluggish recoveries highlights the potential of job retention schemes, like those implemented in Europe during the COVID recession, as an area for future research. As emphasized by \cite{Lachowska2020} and \cite{Jacobson1993}, "something intrinsic to the employment relationship itself... is lost when workers are displaced." Job retention policies may serve as the most effective hedge against scarring, given the inherent challenges of finding a strong employer-employee match. I leave these important questions for future research.


