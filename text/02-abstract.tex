\chap{Abstract}
%
% REPLACE THIS WITH ABSTRACT TEXT

This dissertation is composed of three chapters that study how the experience and fear of unemployment shapes macroeconomic consumption dynamics and evaluate the effectiveness of various fiscal policies designed to stabilize these fluctuations.

The first chapter investigates the macroeconomic consequences of a well documented microeconomic fact: Job loss leaves scars on wages that persist for more than 20 years. This chapter argues that these unemployment `scars' are a key determinant of the speed of macroeconomic recovery following recessions. I incorporate human capital into a heterogeneous agent New Keynesian model with search and matching frictions. During unemployment, human capital depreciates, leading to lower wages for reemployed workers. Unemployment scarring, mediated by the fraction of temporary versus permanent layoffs, enables the model to capture both the sluggish recovery from the Great Recession and the rapid rebound from the COVID Recession. In particular, the presence of scarring reveals the pivotal role that temporary layoffs fulfilled in supporting the swift post-pandemic recovery and in preventing a subsequent permanent rise in inequality. In a counterfactual analysis of the Great Recession, a U.S. fiscal consolidation would have proven substantially less effective at reducing debt-to-GDP as scarring erodes future tax revenues and therefore increases pressure on the fiscal deficit.

The second chapter documents how household beliefs on finding and losing a job evolve over the business cycle. We backcast subjective expectations on job finding and job loss in the Survey of Consumer Expectations to 1978, and use real-time machine learning forecasting to proxy their objective counterparts. We document stickiness in job finding and loss expectations in reflecting changes in real-time job risks and their substantial heterogeneity across observable and unobservable dimensions. Calibration of these facts to a heterogeneous-agent consumption-saving model with countercyclical unemployment risks suggests that a non-trivial size of the drop in aggregate consumption during past recessions stemmed from precautionary responses to heightened perceived job risks, although belief stickiness limited the role of ex-ante self-insurance.

The third chapter evaluates and compares the effectiveness of commonly pursued fiscal stimulus policies during recessions. Using a heterogeneous agent model calibrated to match measured spending dynamics over four years following an income shock, we assess the effectiveness of three fiscal stimulus policies employed during recent recessions. Unemployment insurance (UI) extensions are the clear “bang for the buck” winner when effectiveness is measured in utility terms. Stimulus checks are second best and have two advantages (over UI): they arrive and are spent faster, and they are scalable to any desired size. A temporary (two-year) cut in the rate of wage taxation is considerably less effective than the other policies and has negligible effects in the version of our model without a multiplier.


%\lipsum[1-4]
%
% Add Thesis Readers
\section*{Thesis Readers}
\begin{singlespace}
%
\noindent Dr.~Christopher D. Carroll (Primary Advisor)\\
\indent \indent Professor\\
\indent \indent Department of Economics\\
\indent \indent Johns Hopkins University\\

\noindent Dr.~Jonathan Wright\\
\indent \indent Professor\\
\indent \indent Department of Economics\\
\indent \indent Johns Hopkins University\\

\noindent Dr.~Francesco Bianchi\\
\indent \indent Professor\\
\indent \indent Department of Economics\\
\indent \indent Johns Hopkins University\\

\noindent Dr.~Michael Keane\\
\indent \indent Professor\\
\indent \indent Carey School of Business and Department of Economics\\
\indent \indent Johns Hopkins University\\

\noindent Dr.~Vadim Elenev\\
\indent \indent Associate Professor\\
\indent \indent Carey School of Business\\
\indent \indent Johns Hopkins University\\
%
%\noindent First Lastname \\
%\indent \indent Associate Professor\\
%\indent \indent Affiliation1, and\\
%\indent \indent Department1 \& Department2 at\\
%\indent \indent Johns Hopkins Bloomberg School of Public Health \\
%
\end{singlespace}