\chap{Abstract}
%
% REPLACE THIS WITH ABSTRACT TEXT

This dissertation consists of three chapters that study how the microeconomic characteristics of unemployment shape business cycle dynamics and macroeconomic policy.

The first chapter investigates the macroeconomic consequences of a well documented microeconomic fact: Job loss results in a substantial decline in labor earnings that persists for over 20 years. This chapter argues that this `scarring' effect of unemployment is a key determinant of the speed of a macroeconomic recovery following a recession. I incorporate human capital into a heterogeneous agent New Keynesian model with search and matching frictions. During unemployment, human capital depreciates, leading to lower wages for reemployed workers. Unemployment scarring, mediated by the fraction of temporary versus permanent layoffs, enables the model to capture both the sluggish recovery from the Great Recession and the swift rebound from the COVID Recession. In particular, the presence of scarring reveals the pivotal role that temporary layoffs fulfilled in preventing a sluggish post-pandemic recovery. % In a counterfactual analysis of the Great Recession, a U.S. fiscal consolidation would have proven substantially less effective at reducing debt-to-GDP as scarring erodes future tax revenues and therefore increases pressure on the fiscal deficit.

The second chapter examines how household beliefs about the probability of finding and losing a job evolve over the business cycle. We backcast expectations data on job finding and job loss from the Survey of Consumer Expectations and use real-time machine learning forecasts to construct proxies for their rational counterparts. Our analysis reveals that beliefs about job finding and job loss adjust sluggishly to real-time changes in labor market conditions and exhibit substantial heterogeneity across both observable and unobservable characteristics. We calibrate our empirical findings to a heterogeneous-agent consumption-saving model with persistent unemployment. While belief stickiness dampens the immediate precautionary response in aggregate consumption during a recession, the resulting smaller precautionary buffers slow the recovery by limiting households’ ability to draw down excess savings once conditions improve.



The third chapter evaluates and compares the effectiveness of commonly pursued fiscal stimulus policies during recessions. Using a heterogeneous agent model calibrated to match measured spending dynamics over four years following an income shock, we assess the effectiveness of three fiscal stimulus policies employed during recent recessions. Unemployment insurance (UI) extensions are the clear “bang for the buck” winner when effectiveness is measured in utility terms. Stimulus checks are second best and have two advantages (over UI): they arrive and are spent faster, and they are scalable to any desired size. A temporary (two-year) cut in the rate of wage taxation is considerably less effective than the other policies and has negligible effects in the version of our model without a multiplier.


%\lipsum[1-4]
%
% Add Thesis Readers
\section*{Thesis Readers}
\begin{singlespace}
%
\noindent Dr.~Christopher D. Carroll (Primary Advisor)\\
\indent \indent Professor\\
\indent \indent Department of Economics\\
\indent \indent Johns Hopkins University\\

\noindent Dr.~Jonathan Wright\\
\indent \indent Professor\\
\indent \indent Department of Economics\\
\indent \indent Johns Hopkins University\\

\noindent Dr.~Francesco Bianchi\\
\indent \indent Professor\\
\indent \indent Department of Economics\\
\indent \indent Johns Hopkins University\\

\noindent Dr.~Michael Keane\\
\indent \indent Professor\\
\indent \indent Carey School of Business and Department of Economics\\
\indent \indent Johns Hopkins University\\

\noindent Dr.~Vadim Elenev\\
\indent \indent Associate Professor\\
\indent \indent Carey School of Business\\
\indent \indent Johns Hopkins University\\
%
%\noindent First Lastname \\
%\indent \indent Associate Professor\\
%\indent \indent Affiliation1, and\\
%\indent \indent Department1 \& Department2 at\\
%\indent \indent Johns Hopkins Bloomberg School of Public Health \\
%
\end{singlespace}