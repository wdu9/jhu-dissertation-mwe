
% ----------------------------------------------------------------
\section{Conclusion}
% ----------------------------------------------------------------
More people lose jobs and fewer people find jobs in recessions than in normal times. But do people see these changes coming? This paper asks if business cycle movements in job risks are perceived by the average and heterogeneous households who are exposed to different degrees of job risks. The answer to such a question matters because it affects the relative importance of consumption slump in recessions due to ex-ante heightened risks or unexpected ex-post shocks. This paper finds that the average risk perceptions, primarily those regarding job loss, are slow to reflect the unfolding job risk movements along business cycles, therefore limiting the ex-ante channel in driving consumption response and the degree of self-insurance, resulting in a larger impact by ex-post shock response. Meanwhile, job finding beliefs are less rigid and even overreactive, inducing sizable precautionary saving responses. In addition, the footprints of aggregate market labor conditions are widely heterogeneous, as revealed by substantial heterogeneity in perceived job risks. It is not the average worker, but the marginal one who is particularly exposed to business cycle fluctuations that matter for aggregate demand fluctuations due to counter-cyclical job risks. We show the quantitative importance of aggregate and distributional consumption drop due to precautionary savings, misperceived risks, and unexpected income shock response. 

