
% ----------------------------------------------------------------
\section{Conclusion}
% ----------------------------------------------------------------
%More people lose jobs and fewer people find jobs in recessions than in normal times. But do people see these changes coming? This paper asks if business cycle movements in job risks are perceived by the average and heterogeneous households who are exposed to different degrees of job risks. The answer to such a question matters because it affects the relative importance of consumption slump in recessions due to ex-ante heightened risks or unexpected ex-post shocks. This paper finds that the average risk perceptions, primarily those regarding job loss, are slow to reflect the unfolding job risk movements along business cycles, therefore limiting the ex-ante channel in driving consumption response and the degree of self-insurance, resulting in a larger impact by ex-post shock response. Meanwhile, job finding beliefs are less rigid and even overreactive, inducing sizable precautionary saving responses. In addition, the footprints of aggregate market labor conditions are widely heterogeneous, as revealed by substantial heterogeneity in perceived job risks. It is not the average worker, but the marginal one who is particularly exposed to business cycle fluctuations that matter for aggregate demand fluctuations due to counter-cyclical job risks. We show the quantitative importance of aggregate and distributional consumption drop due to precautionary savings, misperceived risks, and unexpected income shock response. 

Recessions lead to more job losses and fewer job gains—but do households anticipate these shifts in labor market risk? This paper investigates whether business cycle fluctuations in the probability of finding and losing a job are actually perceived by the average household. Understanding the extent to which these risks are anticipated is crucial for distinguishing between consumption declines driven by precautionary behavior and those caused by actual income losses from rising unemployment. The analysis reveals that households’ perceptions of the probability of finding and losing a job adjust slowly to changing economic conditions. Unsurprisingly, this sluggish adjustment in perceived job loss risk diminishes the role of precautionary behavior in driving consumption declines, instead emphasizing the quantitative importance of realized income losses from rising unemployment. However, while a muted precautionary response results in a smaller initial drop in consumption during a recession, it also dampens the recovery, as most households—those who remain employed—have smaller precautionary buffers to draw down once unemployment declines. Household perceptions on the probability finding and losing a job also vary widely across the population, reflecting substantial heterogeneity in how households interpret labor market signals. It is not the average worker, but rather the marginal worker—those most exposed to cyclical job risk—that drives fluctuations in aggregate demand. 

%This paper quantifies the aggregate and distributional consumption effects arising from precautionary savings, misperceived risks, and surprise income shocks.