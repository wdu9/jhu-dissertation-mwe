
% ----------------------------------------------------------------
\section{Introduction}
% ----------------------------------------------------------------
In the state-of-the-art incomplete markets model with search and matching frictions, countercyclical fluctuations unemployment amplifies business cycle fluctuations through two key channels.\footnote{Counter-cyclical idiosyncratic job risks are one of the important drivers of aggregate business cycle fluctuations\citep{bayer2019precautionary,den2018unemployment,broer2021unemployment, graves2020does}. Other papers study the role of unemployment insurance in stabilizing such fluctuations and its distributional impacts\citep{mckay2021optimal,boone2021unemployment,kekre2023unemployment}.} The first is an expectations-driven precautionary channel whereby heightened \textit{fears} of unemployment lead to increased saving and reduced consumption, which in turn depresses aggregate demand. The second is an income channel, where realized income losses from unemployment directly reduce consumption.\footnote{The distinction between ex-ante and ex-post responses is also relevant to the dynamics of durable consumption. \citep{harmenberg2021consumption}}





These two channels are typically disciplined by the observed rate at which workers transition between employment and unemployment. However, the true share of workers moving from employment to unemployment does not necessarily reflect the true ex-ante risk of job loss that dictates a worker's precautionary behavior. Realized separation rates are shaped by unforeseen macroeconomic shocks. For instance, workers in 2019 did not anticipate the COVID-19 pandemic, so their perceived risk of job loss for 2020 was far lower than the actual separation rate observed that year. %As a result, the implications of consumption saving model calibrated to 

Furthermore, the risk of job loss that is perceived by households does not necessarily align with the actual real-time risk of job loss given prevailing macroeconomic conditions. A large literature documents systematic deviations between household expectations and full-information rational expectations (FIRE). This raises a natural question: do households accurately perceive their risk of job loss? If households underreact to rising unemployment risk, they may fail to adequately insure themselves against income shocks, leading to insufficient consumption smoothing. Conversely, an overreaction could trigger a sharp decline in aggregate demand.\footnote{See, for instance, \cite{den2018unemployment}.}



This paper separately measures how (a) perceived unemployment risk, (b) objective unemployment risk, and (c) job transition rates vary over the business cycle, and shows that these measures differ substantially in their cyclical dynamics. The conventional approach to studying expectation formation using survey data relies on a direct comparison between (a) and (c)--i.e., forecast errors--to identify deviations from full-information rational expectations (FIRE). By incorporating measure (b), we can characterize the gap between subjective perceptions of job risk and their ex-ante rational benchmark. This extends existing studies that identify biases in job risk beliefs based solely on ex-post comparisons.\footnote{See, for instance, \cite{stephens2004job}, \cite{spinnewijn2015unemployed}, \cite{mueller2021job}, \cite{balleer2021effects}, etc. } 


In particular, our measure of perceived risk (a) is derived from survey expectations of job risk in the New York Fed’s \emph{Survey of Consumer Expectations} (SCE), which is available only since 2013. To extend the series back to 1978,\footnote{Many series from the Michigan Survey of Consumers begin in 1978.} we employ machine learning algorithms trained on a rich set of expectation-related indicators from the Michigan Survey of Consumers (MSC). We also externally validate our imputation method by confirming that the backcasted versions of several benchmark series by the same procedure align closely with actual observed values. This backcasted series enables us to analyze multiple business cycles and empirically assess the strength of precautionary behavior over a much longer history. 


We create a proxy for (b) using a real-time machine-learning forecast framework following the methodology of \cite{bianchi2022belief}. Specifically, at each point of time in our sample, we perform a LASSO (least absolute shrinkage and selection operator) estimation to select a subset of variables from a set of 600 real-time series of macroeconomic conditions and forward-looking expectations by households and professionals that best predict subsequent labor market flows. We then generate a one-step-ahead forecast using the machine-efficient model that is selected from cross-validation. Real-time predicted job transition rates approximate the best possible risk forecast of the labor markets, hence, serving as a good proxy for the objective ex-ante risks.  



Two main findings emerge from comparing these measures. First, the comparison between (a) perceived unemployment risk and (c) realized job transition rates shows that households’ ex-ante subjective beliefs--especially about job-finding probabilities--are strong predictors of actual labor market transitions. This suggests that individuals form expectations using meaningful private information, consistent with micro-level evidence that workers possess advance signals about their employment prospects.\footnote{See, for instance, \cite{hendren2017knowledge}.}
Second, the comparison between (a) and (b) reveals a systematic gap between subjective beliefs and machine-learning-based forecasts: perceptions respond sluggishly to changes in real-time job risk. While the algorithmic forecasts accurately predict job transitions over a three-month horizon (with the exception of crisis onsets like COVID), average subjective expectations underreact and fail to incorporate available predictive signals indicating a deviation from rational expectations.
         

   

We propose two explanations for why average perceived job risks underreact to real-time macroeconomic labor market conditions. First, information rigidity--households update their beliefs about macroeconomic conditions sluggishly. Second, risk heterogeneity--households face differing levels of job risk, either conditionally or unconditionally, implying that not all households respond equally to aggregate labor market shifts. We find that workers across the distribution of perceived job risks respond to true real-time risks with varying intensity and degrees of stickiness. This underscores the importance of heterogeneity in both actual and perceived job risks over the business cycle. It aligns with a growing body of research showing that heterogeneity in job risk exposure amplifies aggregate demand fluctuations through unemployment risk channels. Since households are unevenly affected by rising job risks in recessions, the unequal mapping from aggregate labor market flows to individual risk perceptions helps explain why average perceived job risks respond less than one-for-one to actual labor market dynamics.




Lastly, we incorporate our measures of perceived and objective unemployment risk, along with observed job transition rates, into a heterogeneous agent model with persistent unemployment. This framework allows us to quantify the extent to which fluctuations in aggregate consumption over the business cycle are driven by precautionary saving versus income losses caused by actual changes in unemployment. We simulate the path of aggregate consumption starting in 1988 under two scenarios. In both, the actual unemployment rate evolves according to observed job transition rates; however, workers' perceptions of job risk differ. In the first scenario, perceptions follow our measure of perceived unemployment risk. In the second, they are aligned with our measure of rational (objective) unemployment risk. Finally, to isolate the precautionary saving channel in each scenario, we simulate a benchmark path of aggregate consumption driven solely by observed job transition rates. The difference between this benchmark and the consumption paths that incorporate workers' perceptions of unemployment risk captures the contribution of precautionary behavior.

Our simulations of aggregate consumption beginning in 1988 show that the precautionary channel is sharp and substantial when workers are assumed to have rational (objective) perceptions of job loss risk. In contrast, when we use workers’ actual risk perceptions—which tend to underreact to macroeconomic dynamics—the strength of the precautionary channel is notably attenuated. Interestingly, this underreaction leads workers to under-insure, resulting in a smaller initial drop in consumption during recessions but a more sluggish recovery afterward, as there is less precautionary savings to draw down. 




We also highlight the important interaction between job risk heterogeneity and belief distortions. Low-educated workers, who are disproportionately exposed to cyclical job risks, exhibit the stickiest beliefs and are therefore the most underinsured when unemployment shocks materialize. This underinsurance amplifies the effects of unemployment risk over the business cycle.\footnote{For example, \cite{patterson2023matching} shows that workers with the most cyclical incomes also have the highest marginal propensities to consume. Similarly, \cite{guerreiro2023belief} identifies the conditions under which the interaction between beliefs, disagreement, and heterogeneity amplifies business cycle dynamics.} Taken together, this evidence suggests that the strength of unemployment and unemployment risk as amplification channels critically depends on how heterogeneous households perceive fluctuations in job risk.










\subsection*{Related Literature}

Our paper builds on the empirical evidence of biases in job-finding expectations as documented by \cite{mueller2021job}, which studies the microdata on job-finding expectations in the SCE. In comparison to their work, we study the job-finding expectations at the macro level. We corroborate their finding by showing that individuals' job-finding expectations underreact to changes in the actual job-finding probability over business cycles, in addition to the underreaction to changes over the unemployment duration. In addition, several other studies based on a comparison of the perceived job risks and realized job transitions, as surveyed in \cite{mueller2023expectations}, provide divergent evidence between over-optimism and over-pessimism in job expectations. For instance, \cite{arni2013s}, \cite{spinnewijn2015unemployed}, \cite{conlon2018labor}, \cite{mueller2021job} all found that workers over-perceive the job-finding probability, with a stronger bias with longer duration of unemployment. \cite{conlon2018labor} shows such bias is due to over-optimism in perceived offer arrival rates and wage offers. \cite{balleer2021effects} explores the consequences of over-optimism bias. Unlike these papers, we primarily focus on the variability of the business cycle fluctuations of these perceptions relative to their realizations, instead of a possibly constant bias. 

On job separation perceptions, \cite{stephens2004job}'s evidence suggests that workers over-perceive the job loss probability compared to the realization. However, the author cautions on the possible selection bias in interpreting this finding, as higher perceived job loss probability might induce workers to opt out of high-risk jobs, lowering the realized job loss probability. The same issue may also be relevant in the scenario of overoptimism in job findings. A few follow-up studies suggest similar upward biases in job loss perceptions.\citep{dickerson2012fears,balleer2023biased} Despite such biases, \cite{dickerson2012fears,hendren2017knowledge,pettinicchi2019job,hartmann2024subjective} suggest that workers' perceived job risks predict the unemployment outcome reasonably well indicating advance information. 

This paper builds on the literature that adopts real-time forecasting to approximate ex-ante uncertainty/risks. This is also closely related to using machine-efficient forecast as the rational benchmark instead of a constructed benchmark under a specific assumption of data-generating process \citep{bianchi2022belief}. Our use of the approach in \cite{bianchi2022belief} is to proximate not just FIRE, but also \emph{ex-ante} job risks. The notion that ex-ante risks are different from ex-post outcomes is also made clear by \cite{jurado2015measuring,rossi2015macroeconomic} in measuring the macroeconomic uncertainty instead of specifically labor income risks.


Our paper directly contributes to several papers that incorporate subjective job risk perceptions in otherwise standard macroeconomic models featuring uninsured job risks. \citep{pappa2023expectations, bardoczy2023unemployment}. In addition, \cite{morales2022cyclical, menzio2022stubborn, rodriguez2023role} incorporate informational frictions in standard search and matching models to resolve the volatility puzzle in the aggregate unemployment rate. Different from their work, we explore the implications of perceived unemployment risks on consumption/saving and aggregate demand fluctuations. Our findings of the heterogeneity in job expectations also relate to \cite{broer2021information}, which relies on endogenous information choice to account for empirical evidence that information frictions in household macroeconomic expectations non-monotonically depend on their wealth. Our finding that rigidity in job beliefs of workers does not often decrease with the cyclical exposure of their job risks, seems to suggest that mechanisms beyond optimal information choices may play a role in causing such belief stickiness. 


