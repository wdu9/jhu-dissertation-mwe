
% ----------------------------------------------------------------
\section{Introduction}
% ----------------------------------------------------------------
In the state-of-the-art macroeconomic model of incomplete markets with search and matching frictions, countercyclical unemployment risk amplifies business cycle fluctuations.\footnote{Counter-cyclical idiosyncratic job risks are one of the important drivers of aggregate business cycle fluctuations\citep{bayer2019precautionary,den2018unemployment,broer2021unemployment, graves2020does}. Other papers study the effect of unemployment insurance in stabilizing such fluctuations and its distributional impacts\citep{mckay2021optimal,boone2021unemployment,kekre2023unemployment}.} This amplification stems from two channels. The first is a precautionary saving channel whereby heightened fears of unemployment dampen consumption, further reducing aggregate demand. The second is an income channel, where a reduction in consumption occurs due to realized income losses from unemployment. \footnote{The distinction between ex-ante and ex-post responses is also relevant to the dynamics of durable consumption. \citep{harmenberg2021consumption}}


Both these channels are typically disciplined by the realized probability of unemployment calculated from microdata. However, given the large body of evidence on how households' macroeconomic expectations deviate from the full-information-rational-expectations (FIRE), it is natural to ask whether perceptions of unemployment risk align with the true probability of losing a job. An underreaction to increased unemployment risk could lead to under-insurance, leaving households vulnerable as they are unable to smooth their consumption in response to shocks. In contrast, an overreaction to increased unemployment risk may induce a substantial fall in aggregate demand.\footnote{See, for instance, \cite{den2018unemployment}.} 

 This paper measures how (a) perceived, (b) objective, and (c) realized unemployment risks evolve over the business cycle. The first two measures capture ex-ante expectations of unemployment risks, while the third is the ex-post realization of the fraction of individuals transitioning into unemployment. Under rational expectations, models of the business cycle featuring countercyclical unemployment risk assume that (a) and (b) are identical. Moreover, the assumption of perfect foresight, commonly used in the empirical implementation of this class of models, implies that (b) and (c) are equivalent. We show that neither assumption proves consistent with the data.



 
In particular, survey expectations of job transition probabilities are used to measure (a), while the method of real-time machine learning forecasting (e.g. \cite{bianchi2022belief}) is used to create a proxy of (b). Separately measuring these two allows us to characterize the difference between subjective job risk perceptions and their ex-ante rational benchmark. The conventional approach of studying expectation formation relies upon a direct comparison of (a) and (c), i.e., the forecast errors, to provide evidence for deviations from FIRE. The existence of ex-post forecast errors, however, does not necessarily imply ex-ante deviations in expectations, as the ex-post realizations may contain realized shocks that could not have been expected even by rational agents. This complements several existing studies that document biases in job beliefs by only comparing (a) and (c).\footnote{See, for instance, \cite{stephens2004job}, \cite{spinnewijn2015unemployed}, \cite{mueller2021job}, \cite{balleer2021effects}, etc. }


% How we overcome problems with a) perceived risk
Although the perceived job risks in survey data directly measure (a), such data is not available until the most recent decade. Therefore, we backcast the series of perceived job risks back to 1978 when there were no directly surveyed beliefs of the same kind. Utilizing the correlation between perceived job risks in New York Fed's \emph{Survey of Consumer Expectations} (SCE) since 2013 and the Michigan Survey of Consumers (MSC) in which numerous other expectations have been measured for a much longer history, we backcast the perceived job risks into the past four decades. This allows our analysis to span multiple business cycles and empirically measure the strength of precautionary motives, circumventing the assumption that ex-post outcomes are equal to ex-ante perceived risks. 

% How we overcome problems with proxy for b)
To measure (b), we adopt a real-time machine-learning forecast framework following the methodology of \cite{bianchi2022belief}. In particular, we generate real-time forecasts of labor market transition rates using numerous real-time variables that include economic conditions, household expectations on the future of the macroeconomy, and personal finance. We also incorporate professional forecasts and other macroeconomic series that may predict subsequent labor market changes. Real-time predicted job transition rates approximate the best possible risk forecast of the labor markets, hence, serving as a good proxy for the objective ex-ante risks.  

         
Specifically, for each point of the time in the sample, we perform a LASSO (least absolute shrinkage and selection operator) estimation to select a subset of variables from a set of 600 time series of real-time macroeconomic conditions and other forward-looking expectations that best predict subsequent labor market flows. We then generate a one-step-ahead forecast using the machine-efficient model. We include survey perceptions of job risks in the prediction model to account for the fact that ex-ante perceptions reported in surveys, although measures of perceptions, turn out to be predictive of ex-post transition rates. This also nests a special case where perceptions perfectly predict ex-post transition rates. Household expectations in MSC are strongly predictive of labor market transition rates, suggesting agents do incorporate useful information in forming expectations on unemployment risk. In addition, not only do real-time conditions correlate with expectations; but also forward-looking economic decisions, such as durable spending intentions. Finally, several series in the MSC provide causal attributions, e.g. not a good time to buy durables because one cannot afford them, are also important predictors. 


%Results of empirical comparison
With direct measures of (a) and (c) and a good proxy for (b), we document two main findings. First, ex-ante subjective job risks, especially regarding job-finding rates, are highly predictive of ex-post job transition rates. This suggests that average households incorporate useful information to form views about their job risks. It is also consistent with the finding in the literature that agents possess advance information about future job transitions. Second, perceived risks do not perfectly coincide with machine-efficient forecasts in that the former are upward biased and underreactive to the changes in the latter. Machine-efficient forecasts produced by the aforementioned procedure are found to be highly accurate in forecasting labor transition rates in the 3-month horizon. Average subjective perceptions of job risks, despite their predictive power, do not fully update synchronously with rational ex-ante risks, suggesting that they fail to efficiently incorporate all the information that predicts subsequent labor market changes. 

The sluggish response in perceived job risks limits the ex-ante precautionary saving channel's role while amplifying the ex-post shock response channel. In addition, there is an important difference between normal times and crisis episodes in terms of the relative importance of two contributors of ex-post channel, one from misperceived risks, namely the gap between (a) and (b), and the truly unexpected unemployment shocks, namely the gap between (b) and (c). During normal times, the former was the key. This means households do not see the risks that are actually already unfolding and are therefore under-prepared when the unemployment happens. In a few crisis episodes such as the outbreak of the COVID-19 crisis, in contrast, it is the latter that matters more. The sudden increase in unemployment was a truly unexpected shock and could not have been perceived ex-ante even by the most informed forecasters in the economy.      

We provide two explanations for why average perceived risks underreact to real-time macroeconomic labor risks. The first is information rigidity, in that households sluggishly learn about macroeconomic conditions. The second is heterogeneity, in that households face either conditional or unconditional heterogeneity in job risks. This implies that households do not need to react equally to aggregate labor market conditions. In particular, we find that workers across the distribution of perceived job risks react to true real-time risks by different degrees and exhibit various biases. This highlights the role of heterogeneity in true and perceived job risks workers face over business cycles. It is consistent with an increasing number of studies that emphasize the role of heterogeneity in job risks in amplifying aggregate demand fluctuations via unemployment risk channels.\footnote{For instance, \cite{patterson2023matching} shows that the group of workers whose income has the largest cyclical movements also have high marginal propensities to consume. } Households are unevenly affected by increasing job risks in recessions. The heterogeneity in the effects of aggregate labor market flow rates on individual job risks, therefore, helps explain why average perceived job risks do not one-to-one react to the true real-time job risks. 

Lastly, we quantify the aggregate demand fluctuations due to unemployment and unemployment risk allowing for sticky and heterogeneous risk perceptions in a standard Heterogeneous-agent model with persistent unemployment. Our empirical measures of perceptions and outcomes tell their time-series volatility per se, but it is together with the heterogeneous households' consumption/saving sensitivity with respect to risks and shocks that governs the degree of aggregate demand fluctuations. We therefore decompose the aggregate consumption Jacobians (in the terminology of \cite{auclert2021using}) to a given shock of future job-separation and finding probability onto the ex-ante precautionary response given perceptions of such a shock, under-insurance due to misperceived risk, and ex-post shock responses. Then the decomposed Jacobians are combined with the empirically estimated shocks to perceived risk,  objective risk, and realized job transitions to quantify the consumption impacts of these three channels.  The second channel largely contributes to the ex-post drop in consumption. We show that allowing for subjective and heterogeneous perceptions of risks yields a more persistent drop in aggregate consumption during recessions than a model assuming rational expectations and perfect foresight. This result suggests that the strength of unemployment and unemployment risk channels in amplifying business cycle fluctuations crucially depend on how the heterogeneous households perceive the fluctuations in job risks. 


\subsection*{Related Literature}

Our paper builds on the empirical evidence of biases in job-finding expectations as documented by \cite{mueller2021job}, which studies the microdata on job-finding expectations in the SCE. In comparison to their work, we study the job-finding expectations at the macro level. We corroborate their finding by showing that individuals' job-finding expectations underreact to changes in the actual job-finding probability over business cycles, in addition to the underreaction to changes over the unemployment duration. In addition, several other studies based on a comparison of the perceived job risks and realized job transitions, as surveyed in \cite{mueller2023expectations}, provide divergent evidence between over-optimism and over-pessimism in job expectations. For instance, \cite{arni2013s}, \cite{spinnewijn2015unemployed}, \cite{conlon2018labor}, \cite{mueller2021job} all found that workers over-perceive the job-finding probability, with a stronger bias with longer duration of unemployment. \cite{conlon2018labor} shows such bias is due to over-optimism in perceived offer arrival rates and wage offers. \cite{balleer2021effects} explores the consequences of over-optimism bias. Unlike these papers, we primarily focus on the variability of the business cycle fluctuations of these perceptions relative to their realizations, instead of a possibly constant bias. 

On job separation perceptions, \cite{stephens2004job}'s evidence suggests that workers over-perceive the job loss probability compared to the realization. However, the author cautions on the possible selection bias in interpreting this finding, as higher perceived job loss probability might induce workers to opt out of high-risk jobs, lowering the realized job loss probability. The same issue may also be relevant in the scenario of overoptimism in job findings. A few follow-up studies suggest similar upward biases in job loss perceptions.\citep{dickerson2012fears,balleer2023biased} Despite such biases, \cite{dickerson2012fears,hendren2017knowledge,pettinicchi2019job,hartmann2024subjective} suggest that workers' perceived job risks predict the unemployment outcome reasonably well indicating advance information. 

This paper builds on the literature that adopts real-time forecasting to approximate ex-ante uncertainty/risks. This is also closely related to using machine-efficient forecast as the rational benchmark instead of a constructed benchmark under a specific assumption of data-generating process \citep{bianchi2022belief}. Our use of the approach in \cite{bianchi2022belief} is to proximate not just FIRE, but also \emph{ex-ante} job risks. The notion that ex-ante risks are different from ex-post outcomes is also made clear by \cite{jurado2015measuring,rossi2015macroeconomic} in measuring the macroeconomic uncertainty instead of specifically labor income risks.


Our paper directly contributes to several papers that incorporate subjective job risk perceptions in otherwise standard macroeconomic models featuring uninsured job risks. \citep{pappa2023expectations, bardoczy2023unemployment}. In addition, \cite{morales2022cyclical, menzio2022stubborn, rodriguez2023role} incorporate informational frictions in standard search and matching models to resolve the volatility puzzle in the aggregate unemployment rate. Different from their work, we explore the implications of perceived unemployment risks on consumption/saving and aggregate demand fluctuations. Our findings of the heterogeneity in job expectations also relate to \cite{broer2021information}, which relies on endogenous information choice to account for empirical evidence that information frictions in household macroeconomic expectations non-monotonically depend on their wealth. Our finding that rigidity in job beliefs of workers does not often decrease with the cyclical exposure of their job risks, seems to suggest that mechanisms beyond optimal information choices may play a role in causing such belief stickiness. 


